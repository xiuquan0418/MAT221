\documentclass{exam}
\usepackage{amsmath,enumerate,tikz,xcolor,geometry}
\pagestyle{empty}
\textwidth16cm
\textheight23cm
\oddsidemargin0pt
\topmargin-50pt
\def\R{\rule[-1ex]{0ex}{3.5ex}}
\def\ans#1{\hfill \textcolor{lightgray}{#1}}

\begin{document}
\noindent
Tougaloo College\hfill MATH 221 \\
Calculus I \hfill 
\begin{center}
{\bf Practice Test 1}
\end{center}

\medskip
\noindent
Name:\underline{\hspace{3in}} \hfill Score:\underline{\hspace{0.8in}}

\medskip
\begin{enumerate}
\item
Assuming $f(x)=x^{2}$, evaluate 
$\frac{f(x+h)-f(x)}{h}$
\vfill

%\item
%If $f(x)=x^{3}-2x+1$,  then $f$ is the function $f$ even, odd, or neither?%\ans{(sec. 1.6 \#40)}
%
%\vfill

\item Let $f(x) = 2+ \frac{1}{x+3}.$ Determine the inverse function of $f$, which we write as $f^{-1}$.
\vfill



\item
Find the domain of the following function
\begin{parts}
\part $f(x)=\frac{2x+1}{x^{2}+x-2}$%
\vfill
\part $g(x)=\sqrt{3x-5}$
\vfill
\part $h(x)=\frac{\sqrt{2x^{2}+1}}{3x-5}$
\vfill
\end{parts}

\newpage
\item
Given 
$  f(x)=\left\{
\begin{array}{ll}
      1-x^{2},& x\leq 0\\
      2x+1,   & x>0 \\
\end{array} 
\right. $
\begin{parts}
\part Evaluate $f(-2)$ and $f(1)$%
\vfill
\part Sketch the graph of the function $f(x)$%
\vfill
\end{parts}

\item
If $f(x)=x^{2}+2x-1$ and $g(x)=2x-3$, find and simplify $\left[f\circ g\right](x)$ and $\left[g\circ f\right](x)$

\vfill




\newpage
%\item
%let $f(u)=u^{2}+\frac{1}{u+1}$, and $g(x)=x^{4}+x^{2}$. Find $\left(f\circ g\right)(x)$ and Dom$\left(f\circ g\right)$%\ans{(sec. 1.6 \#40)}
%
%\vfill

\item
Find the value of the limit, and, when applicable, indicate the limit theorems being used.  %\ans{(sec. 0.8 \#15)}

\begin{minipage}{.5\linewidth}
\begin{equation*}
 \lim_{x\to3} (2x^{2}-4x+5)
\end{equation*}
\end{minipage}%
\begin{minipage}{.5\linewidth}
\begin{equation*}
 \lim_{x\to 3} \frac{2x^{2}-5x-3}{x-3}
\end{equation*}
\end{minipage}



\vfill

\begin{minipage}{.5\linewidth}
\begin{equation*}
 \lim_{x\to 2} \frac{x^{4}-16}{x-2}
\end{equation*}
\end{minipage}%
\begin{minipage}{.5\linewidth}
\begin{equation*}
\lim_{x\to 4} \frac{\sqrt{x}-2}{x-4}
\end{equation*}
\end{minipage}
\vfill


\begin{minipage}{.5\linewidth}
\begin{equation*}
 \lim_{x\to0} \frac{\sqrt{x+2}-\sqrt{2}}{x}
\end{equation*}
\end{minipage}%
\begin{minipage}{.5\linewidth}
\begin{equation*}
 \lim_{x\to0} \frac{1-\cos{x}}{\sin{x}}
\end{equation*}
\end{minipage}
\vfill


\newpage


\begin{minipage}{.5\linewidth}
\begin{equation*}
 \lim_{x\to 3} \frac{\frac{1}{x^{2}}-\frac{1}{9}}{x-3}
\end{equation*}
\end{minipage}%
\begin{minipage}{.5\linewidth}
\begin{equation*}
\lim_{x\to \frac{1}{2}} \frac{x^{-1}-2}{x-\frac{1}{2}}
\end{equation*}
\end{minipage}
\vfill





\begin{minipage}{.5\linewidth}
\begin{equation*}
 \lim_{x\to \infty} \frac{2x^{4}-x+1}{5x^{4}+x^3}
\end{equation*}
\end{minipage}%
\begin{minipage}{.5\linewidth}
\begin{equation*}
\lim_{x\to \infty} \sin{x}
\end{equation*}
\end{minipage}
\vfill



\begin{minipage}{.5\linewidth}
\begin{equation*}
 \lim_{x\to 0^{-}} e^{\frac{1}{x}}
\end{equation*}
\end{minipage}%
\begin{minipage}{.5\linewidth}
\begin{equation*}
 \lim_{h\to 0} \frac{(4+h)^{2}-16}{h} 
\end{equation*}
\end{minipage}
\vfill








%%%%%%%%%%%%%%%%%%%%%%%%%%%%%%%%%%%%%%%%%%%%%%%%%%%%%%%%-----New Page-----%%%%%%%%%%%%%%%%%%%%%%%%%%%%%%%%%%%%%%%%%%%
\newpage

\item
Given that $\lim_{x\to c} f(x)=-3$, $\lim_{x\to c} g(x)=0$ and $\lim_{x\to c} h(x)=8$, find the following limit. If the limit does not exist, write $\bf{DNE}$ . 
\\
\\
\begin{minipage}{.5\linewidth}
\begin{equation*}
a).  \lim_{x\to c} (f(x)+h(x))
\end{equation*}
\end{minipage}%
\begin{minipage}{.5\linewidth}
\begin{equation*}
b).  \lim_{x\to c} \frac{2f(x)}{h(x)-f(x)}
\end{equation*}
\end{minipage}
\\
\\
\begin{minipage}{.5\linewidth}
\begin{equation*}
c).  \lim_{x\to c} \frac{f(x)}{g(x)}
\end{equation*}
\end{minipage}%
\begin{minipage}{.5\linewidth}
\begin{equation*}
d).  \lim_{x\to c} \frac{g(x)}{f(x)}
\end{equation*}
\end{minipage}
\vfill








\item
Let 
\[
f(x)=\frac{x^{2}-9}{|x-3|}
\]

Find each limit (if it exists).

\begin{minipage}{.3\linewidth}
\begin{equation*}
 \lim_{x\to 3^{-}}f(x),
\end{equation*}
\end{minipage}%
\begin{minipage}{.3\linewidth}
\begin{equation*}
 \lim_{x\to 3^{+}}f(x),
\end{equation*}
\end{minipage}%
\begin{minipage}{.3\linewidth}
\begin{equation*}
 \lim_{x\to 3}f(x)
\end{equation*}
\end{minipage}

\vfill
\vfill

\item
Let 
\[   f(x)=\left\{
\begin{array}{ll}
      x+2, & x\leq -1 \\
      x^{2}-1,& -1\leq x < 2\\
      \sqrt{x+1},   & x>2 \\
\end{array} 
\right. \]

Find each limit (if it exists).

\begin{minipage}{.3\linewidth}
\begin{equation*}
 \lim_{x\to -1^{-}}f(x),
\end{equation*}
\end{minipage}%
\begin{minipage}{.3\linewidth}
\begin{equation*}
 \lim_{x\to -1^{+}}f(x),
\end{equation*}
\end{minipage}%
\begin{minipage}{.3\linewidth}
\begin{equation*}
 \lim_{x\to -1}f(x)
\end{equation*}
\end{minipage}


\vfill

\begin{minipage}{.3\linewidth}
\begin{equation*}
 \lim_{x\to 2^{-}}f(x),
\end{equation*}
\end{minipage}%
\begin{minipage}{.2\linewidth}
\begin{equation*}
 \lim_{x\to 2^{+}}f(x),
\end{equation*}
\end{minipage}%
\begin{minipage}{.2\linewidth}
\begin{equation*}
 \lim_{x\to 2}f(x)
\end{equation*}
\end{minipage}
\begin{minipage}{.2\linewidth}
\begin{equation*}
 \lim_{x\to 3}f(x)
\end{equation*}
\end{minipage}
\vfill

%%%%%%%%%%%%%%%%%%%%%%%%%%%%%%%%%%%%%%%%%%%%%%%%%%%%%%%%-----New Page-----%%%%%%%%%%%%%%%%%%%%%%%%%%%%%%%%%%%%%%%%%%%
\newpage



\vfill

\item
Given
$f(x)=\left\{
\begin{array}{ll}
      cx^{2}, & x < 1 \\
      4     , & x = 1 \\
      -x^{3}+mx,   & x > 1 \\
\end{array} 
\right.$.\\
Find the value $c$ and $m$, so that the function is continuous at $x=1$
\vfill

\item
use the Squeeze Theorem to find the following limit
\begin{parts}
\part If $1\leq f(x)\leq x^{2}+2x+2$, find  $\lim_{x\to -1} f(x)$
\vfill
\part Given $g(x)=x^{2}\sin{\frac{1}{x}}$, find $\lim_{x\to 0}g(x)$
\vfill
\end{parts}



\newpage


\item
Prove that $x^{2}+x-1=11$ has a solution on $[0, 5]$.
\vfill

%\item Use the definition of the limit to prove the following limit. 
%\[
%\lim_{x\to 2}(5x-4)=6
%\]
%\vfill


\end{enumerate}
\end{document}

